\section{Ideas}

本次Project分为两部分,下面就这两部分分别进行讨论:

\subsection{expr$\_$to$\_$truthtable}
	首先,通过将题目抽象,这一部分是一个经典的表达式求值的问题,其主要问题如下:
		$$\mbox{利用算符优先关系,实现对给定混合运算表达式的求值。}$$
		
	这是个栈的经典应用问题,常见解法是将中缀表达式(即输入的表达式)转化为后缀表达式后求解。时间复杂度为$O(n)$,其中$n$为变量的个数
	
	具体的算法会在main$\_$algorithm模块详细介绍。
	
	表达式求值是一个非常经典的问题,已经有了许多成熟的解法。甚至在Python中,一条evaluate语句就可以完成这样的工作。但是对于含字母的表达式,至今没有太好的解决办法。
	
	但是考虑到本次作业的特殊性,一共只有最多8个变量,同时是逻辑计算而不是算术计算,这无疑极大的降低了难度。对于普通的算术含参表达式,我们只能多次取不同的大素数代入变量,还不能保证正确性。但是在本题的布尔代数式中,一共只有$2 ^ 8 = 256$种取值方式,是完全可以接受的。于是,我们可以先遍历所有变量的可能取值,代入表达式中,从而计算出真值表中对应的值。
	
	因此,总的时间复杂度为$O(n * 2 ^ n)$, 由于 $n \le 8$,复杂度不会超过$8 * 2 ^ 8 = 2048$,是完全可以接受的。
\subsection{truthtable$\_$to$\_$expr}

	对于将真值表转化为表达式的问题,我们在数字逻辑电路中曾学习过使用卡诺图的方法,非常直观,同时在对于维数较低时也易于化简。
	
	但是卡诺图的解题过程较为繁琐,同时对于高维的情况扩展性较差。在$n \geq 6$时,就比较难画出合适的卡诺图了。因此,我们选用Quine-McCluskey算法。它虽然较为抽象,但是非常模式化,对高维的扩展性很好。
	
	至于算法的时间复杂度方面,由于化简表达式至今还是NP-hard问题,因此只存在指数级解法,即问题的时间复杂度为$O(2 ^ n)$
	
	同时,题目要求将表达式化到最简形式,因此在Quine-McCluskey算法后我们必须进行化简。这里采用petrick化简算法,时间复杂度同样为$O(2 ^ n)$
	
	具体的算法会在main$\_$algorithm模块详细介绍。

\subsection{异常处理}
	我还没做呢