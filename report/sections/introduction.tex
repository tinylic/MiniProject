
\section{Overview}

\subsection{问题描述}

\begin{itemize}

\item{\Large{编写一个函数,计算逻辑表达式的真值表}}
	\begin{itemize}
		\item	逻辑表达式最多有8个输入项,分别为$A,B,C,D,E,F,G,H$;

		\item	支持的逻辑运算符按优先级从高到低分别为$\sim$(非)、$\&$(与)、$\land$(异或)、$\mid$(或);
	
	\end{itemize}
\item{\Large{编写一个函数,根据真值表计算逻辑表达式,以字符串格式输出}}
	\begin{itemize}
		\item	使用Quine-McCluskey算法对逻辑表达式进行化简。	
	\end{itemize}

\item{\Large{函数接口:}}
	\begin{itemize}
		\item	std::string expr$\_$to$\_$truthtable(int n, const std::string$\&$ expr);

其中n是变量个数,expr是逻辑表达式,返回真值表。

		\item	std::string truthtable$\_$to$\_$expr(const std::string$\&$ truth$\_$table);

其中truth$\_$table是真值表,返回逻辑表达式。
	
	\end{itemize}

若参数无效,函数抛出异常。

\item{\Large{要求}}
	\begin{itemize}
		\item	使用第5讲的测试框架对两个函数进行测试;提交源代码与设计文档。
		\item	文档内容包括:设计思路、数据结构与算法、重要的类与函数的说明、测试用例的设计等等,也可以写完成Project的心得体会、经验与教训等。
	
	\end{itemize}

\end{itemize}

\subsection{解题概况}
	最终程序各部分及其作用如下表

	\begin{center}
		\begin{tabular}{|c|c|}
			\hline  Documents & Functions\\
			\hline  constants.h & 常量的定义\\
			\hline  simple$\_$test.h & 参与测试\\
			\hline  ChartToExpression.h & 将真值表转换为逻辑表达式\\
			\hline  ExpressionToChart.h & 将逻辑表达式转换为真值表\\
			\hline  implication & 质蕴涵项的实现\\
			\hline
		\end{tabular}
	\end{center}	
	
	下面就各部分的解题思路及实现进行详细讨论	