\section{Project Structures}
\subsection{ChartToExpression}

对于第一个问题,我们使用ChartToExpression类来实现,程序声明如下:

\lstinputlisting{sources/ExpressionToChart.txt}

其中各变量及函数作用如下:

\begin{itemize}
	\item{变量}
	\begin{center}
		\begin{tabular}{cc}
			变量 & 作用 \\
			operators[] & 储存定义的四种运算符\\
			Stack$\_$Operator & 运算符的堆栈 \\
			Stack$\_$Number & 运算结果的堆栈 \\
			loc[MAX$\_$N] & 储存字符出现位置,便于代入数值\\
		\end{tabular}
	\end{center}
	\item{函数}
	\begin{center}
		\begin{tabular}{cc}
			函数 & 作用 \\
			ExpressionToChart() & 构造函数\\
			virtual ~ExpressionToChart() & 析构函数 \\
			Stack$\_$Number & 运算结果的堆栈 \\
			string solve(int n, const string $\&$InString) & 
			\textbf{将输入的表达式转换为真值表输出}\\
			string filter(const string $\&$s) &
			将输入的表达式去除多余空格\\
			string InfixToPostfix(const string $\&$infix) &
			\textbf{中缀表达式转换为波兰式}\\
			void PushString(string $\&$s, const char $\&$ch) &
			转化时向波兰式中添加字符\\
			bool IsOperator(char ch) &
			判断当前字符是否为运算符\\
			int priority(char Operator) &
			若是运算符,返回其优先级 \\
			int SolvePostfix(const string $\&$postfix) &
			\textbf{计算后缀表达式的值}\\
			void GetTwoNumbers(bool $\&$first, bool $\&$second) &
			在运算结果栈中取出两个元素用于计算\\
			bool CalcEq(bool first, bool second, char op) &
			给定运算数和运算符,计算结果\\
		\end{tabular}
	\end{center}
\end{itemize}
\subsection{ExpressionToChart}
对于将真值表转换为表达式的问题,我们采用Quine-McCluskey算法,主要使用ExpressionToChart类实现。

同时,由于对质蕴涵项这一概念使用较多,因此将它单独建立implication类,以增强内聚度

implication类声明如下:
\lstinputlisting{sources/implication.txt}

其中各变量及函数作用如下:

\begin{itemize}
	\item{变量}
	\begin{center}
		\begin{tabular}{cc}
			变量 & 作用 \\
			int TotalVariables & 总变量个数\\
			int bit & 当前质蕴涵的二进制表示\\
			int xterm & 质蕴涵的任意项,该位为1即为任意项 \\
			int ones & 质蕴涵中1的个数 \\
			string exp & 当前质蕴涵的二进制表示(输出用)\\
			bool used & 表示该蕴涵项是否被覆盖(Q-M算法中)\\
			bool selected & 表示该蕴含项是否是出现在结果中(petrick化简) \\
			vector<int> ImpContained & 该质蕴涵包含的蕴涵项\\

		\end{tabular}
	\end{center}
	\item{函数}
	\begin{center}
		\begin{tabular}{cc}
			函数 & 作用 \\
			implication() & 默认构造函数\\
			virtual ~implication() & 析构函数 \\
			implication(int mask, int xterm, int TotalVariables) & 构造函数\\
			int CountOne(int x) & 统计x的二进制表示中1的个数\\
			string show(); & 输出该蕴涵项\\
		\end{tabular}
	\end{center}
\end{itemize}


ChartToExpression类声明如下:
\lstinputlisting{sources/ChartToExpression.txt}

其中各变量及函数作用如下:

\begin{itemize}
	\item{变量}
	\begin{center}
		\begin{tabular}{cc}
			变量 & 作用 \\
			int TotalVariables & 总变量个数 \\
			vector<int> MinTerm & 输入的最小项,用int表示\\
			vector<implication> imp & 储存最小项,参与Q-M算法计算\\
			vector<implication> roller & Q-M算法中与imp迭代的数组\\
			bool table[1 << MAX$\_$N][1 << MAX$\_$N] &
			可视化蕴涵与最小项的关系\\
			vector<implication> primes &
			储存质蕴涵项\\

		\end{tabular}
	\end{center}
	\item{函数}
	\begin{center}
		\begin{tabular}{cc}
			函数 & 作用 \\
			ChartToExpression() & 构造函数\\
			virtual ~ChartToExpression() & 析构函数\\
			string solve(const string $\&$truth$\_$table) &
			输入真值表,返回化简后表达式\\
			void Quine-McCluskey() & 实现Quine-McCluskey算法\\
			void Simplify() & 对结果化简,使用到了petrick算法\\
			int CountOne(int x) & 统计x的二进制表示中1的个数\\
			bool CheckContained(const implication $\&$imp, const int $\&$x) &
			判断蕴涵Imp是否包含了最小项x\\
			void ShowTable() & 将中间结果可视化\\
		\end{tabular}
	\end{center}
\end{itemize}

